\documentclass{article}

\usepackage{amsmath}
\usepackage{graphicx}
\usepackage{subcaption}

\title{Cardano's Formula for Cubic Equations}
\author{Ananmeti Zion Umoh}
\date{November 18, 2021}

\begin{document}
	\pagenumbering{gobble}
	\maketitle
	\pagenumbering{arabic}
	\begin{center}
		\textbf{Abstract}
	\end{center}
\begin{center}

	Gerolamo Cardano was born in Pavia in 1501 as the illegitimate\\
	child of a jurist. He attended the University of Padua and became\\
	a physician in the town of Sacro, after being rejected by his home
	town of Milan. He became one of the most famous doctors in all of\\
	Europe , having treated the Pope. he was also an astrolger and an\\
	avid gambler, to which he wrote the Book on Games of Chnace, which was th first serious treatise on the mathematics of probabiltiy [1].\\
\end{center}
\section{\textbf{Introduction to Cardano's Formula}}

Cardano's formula for solution of cubic equations for an equation like;
\begin{align*}
	& x^2 + a_1x^2 + a_2x+a^3 = 0\\
\end{align*}
the parameter Q, R, S and T can be computed thus,
\begin{align*}
	&Q=\frac{3a_2-a^2}{a}
	&R=\frac{9a_1a_2-27a^3-2a^3}{54}
\end{align*}
\begin{align*}
	&S=3\sqrt{R+\sqrt{-Q^3+R^2}}
	&T=\sqrt{R-\sqrt{-Q^3+R^2}}
\end{align*}
to give the roots;
\begin{align*}
	&x_1=S+T-\frac{1}{3}a_1\\
	&x_2=\frac{-(S+T)}{2}-\frac{a_1}{3}+i\frac{\sqrt{3}(s-T)}{2}-\frac{a_!}{3}+i\frac{\sqrt{3}(s-T)}{2}\\	&x_2=\frac{-(S+T)}{2}-\frac{a_1}{3}+i\frac{\sqrt{3}(s-T)}{2}-\frac{a_!}{3}+i\frac{\sqrt{3}(s-T)}{2}
\end{align*}
\begin{align*}
	Note: &x_3 must not have a co-efficient.
\end{align*}
\subsection{Some Examples}
\begin{enumerate}
			\item &x^3-3x^2+4=0
			\item &2x^3+6x^2+1=0
\end{enumerate}

\paragraph{\textbf{References}}
|1| P. Scarani, "Out of control: vita di gerolamo cardano (1501-1576),"
Pathologica, vol. 93, pp. 565-574, 2001.



\end{document}