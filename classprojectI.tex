\documentclass{article}

\usepackage{graphicx}
\usepackage{booktabs}
\usepackage{array}
\usepackage{multirow}
\usepackage{siunitx}
\usepackage{pgfplotstable}
\sisetup{
	round-mode= places,
	round-precision= 2,
}

\begin{document}
\begin{table}[h!]
	\begin{center}
		\caption{Classification of Robots}
		\label{table1}
		\pgfplotstabletypeset[
		multicolumn names, % allows to have multicolumn names
		col sep=comma, %the separator in our .csv file
		display columns/0/.style={
			column name=$Source$,
			column type={S},string type},
		display columns/1/.style={
			column name=$Num 2$,
			column type={S},string type},
		display columns/2/.style={
			column name=$Num 3$,
			column type={S},string type},
		display columns/3/.style={
			column name=$Num 4$,
			column type={S},string type},
		display columns/4/.style={
			column name=$Num 5$,
			column type={S},string type},
		display columns/5/.style={
			column name=$Num 6$,
			column type={S},string type},
		every head row/.style={
			before row={\toprule}, % have a rule at top
			after row={
				\si{\ampere} & \si{\volt}\\ % the units separated by &
				\midrule} % rule under units
		},
		every last row/.style={after row=\bottomrule}, % rule at bottom
		]{Robots.csv} % filename/path to file
	\end{center}
\end{table}
\end{document}